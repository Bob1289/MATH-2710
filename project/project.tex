\documentclass{article}
\usepackage[utf8]{inputenc}

\title{Solution to a Nonzero Matrix}
\author{Benny Chen}
\date{\today}

\usepackage{color}
\usepackage{amsthm}
\usepackage{amssymb} 
\usepackage{amsmath}
\usepackage{listings}
\usepackage{xcolor}
\usepackage{listings}
\usepackage{graphicx}
\usepackage[hidelinks]{hyperref}

\definecolor{codegreen}{rgb}{0,0.6,0}
\definecolor{codegray}{rgb}{0.5,0.5,0.5}
\definecolor{codepurple}{rgb}{0.58,0,0.82}
\definecolor{backcolour}{rgb}{0.95,0.95,0.92}

\lstdefinestyle{mystyle}{
    backgroundcolor=\color{backcolour},   
    commentstyle=\color{codegreen},
    keywordstyle=\color{magenta},
    numberstyle=\tiny\color{codegray},
    stringstyle=\color{codepurple},
    basicstyle=\ttfamily\footnotesize,
    breakatwhitespace=false,         
    breaklines=true,                 
    captionpos=b,                    
    keepspaces=true,                 
    numbers=left,                    
    numbersep=5pt,                  
    showspaces=false,                
    showstringspaces=false,
    showtabs=false,                  
    tabsize=2
}

\lstset{style=mystyle}

\begin{document}

\maketitle

\section{Introduction}
In this paper, we will be proving that a linear system of equations with real coefficients has a unique solution if and only if the determinant of its associated matrix $A$ is nonzero. 
We will do this by proving the following theorem:
\\\\
If the determinant of a matrix A is nonzero, then the system of equations $Ax = b$ has a unique solution.

\section{Proof}

Theorem: Let $A$ be a $n \times n$ matrix with nonzero determinant. Then, the system of equations $Ax = b$ has a unique solution.

Let 
\begin{equation}
    A \begin{bmatrix}
    x_1 \\
    x_2 \\
    x_3 \\
    \vdots \\
    \end{bmatrix}
    = \begin{bmatrix}
    b_1 \\
    b_2 \\
    b_3 \\
    \vdots \\
    \end{bmatrix}
\end{equation}
be a system of equations with real coefficients. 
\\\\
Let $A$ be a $n \times n$ matrix with nonzero determinant. 
\\\\
Let $x$ be a vector in $\mathbb{R}^n$ such that $Ax = b$. We will prove that $x$ is the unique solution to the system of equations $Ax = b$.
\\\\
We find if $det(A)$ is invertible.
\\\\
Suppose that $M$ is invertible and $det(M) = 0$
\\\\
Then there exists a matrix $B$ such that $B = M^{-1}$
Which means that:
\begin{equation}
    MB = I
\end{equation}
Where $I$ is the identity matrix. We can then solve for:
\begin{equation}
    det(MB) = det(I)
\end{equation}
Due to the fact that $det(I) = 1$, we can then solve for $det(MB)$:
\begin{equation}
    det(B)0 = 0
\end{equation}
We can now see a that a contradiction has been made
\begin{equation}
    0 = 1
\end{equation}
Therefore a invertible matrix cannot have an determinant of 0.
\\\\
Since we know that $A$ must be invertible now, we solve for $Ax = b$ by left multipling both sides by $A^{-1}$ to get:
\begin{equation}
    A^{-1}Ax = A^{-1}b
\end{equation}
This means that:
\begin{equation}
    x = A^{-1}b
\end{equation}
We can now the original equation $Ax = b$ to get:
\begin{equation}
    A(A^{-1}b) = b
\end{equation}
With $AA^{-1} = I$, we can then solve for $b$:
\begin{equation}
    I * b = b
\end{equation}
So, 
\begin{equation}
    b = b
\end{equation}
So, $x = A^{-1}b$ is a solution to the system of equations $Ax = b$.
\\\\
We can now prove that $x$ is the unique solution to the system of equations $Ax = b$.
\\\\
Let $x$ be a solution to the system of equations. Then,
\begin{equation}
    Ax = b
\end{equation}
So then,
\begin{equation}
    A^{-1}Ax = A^{-1}b
\end{equation}
Which shows that:
\begin{equation}
    x = A^{-1}b
\end{equation}
Therefor, this shows that whenever there is a solution to the system of equations, then $x$ is the only unique solution to the system of equations.


\end{document}